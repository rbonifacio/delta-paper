\section{Delta-Oriented Programming}

% apresentar delta de uma maneira um pouco mais
% formal inicialmente, se concentrando no artigo:
% Delta-Oriented programming of software product lines.
% http://link.springer.com/chapter/10.1007%2F978-3-642-15579-6_6

DOP, Delta Oriented Programming, is a novel programming language approach focused on  modularity, expressiveness and flexibility for the development of software product lines ~\cite{}. DOP uses the concept of delta modules, which are units of modularity that specifies changes to be applied to a core module (minimal implementation of the mandatory features of a product) in a partial order.
A core module is a valid product build using tradicional single application engenieering techiniques. The concept of core module was conceived in a proof-of-concept implementation of DOP, called Core Delta, later version the need for a pre defined core module was abandoned (Pure Delta) ~\cite{}.

\subsection{DeltaJava and Core DOP}
DeltaJava is a DOP is not restricted to any specific programming language, and was   tested in Java, in a tool named DeltaJava. Deltas in DeltaJava can augment, modify or remove code from core deltas modules, on the class level or class structure level ~\cite{}. 

Delta modules are similar to features modules in way it is possible to add or modify the beahvior of a base product, but DOP extends the capabilities of FOP by a operator able to remove code. 

% TODO:
% Product generation
% Safe program generation
% DOP vs FOP - expressiviness, domain feature, optimal feature problem, safe composition and evolutions}

\subsection{DeltaJ 1.5 and Pure DOP}
% DeltaJ 1.5
% Core DOP Vs Pure DOP
% Full Java 1.5
% New language for product line declaration



% essa secao pode ser subdividida em outras subsecoes,
% em particular uma que descreve DeltaJ 1.5
