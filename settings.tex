\section{Study Settings}

% Discutir a organizacao geral do nosso
% estudo de caso, em particular qual o objetivo
% que temos. Em termos de um GQM, aqui poderia
% ser incluido o objetivo e as questoes de pesquisa. 

% ver artigos:
% 
% - http://people.irisa.fr/Edward-Mauricio.Alferez_Salinas/REJ/10.1007_s00766-013-0184-5.pdf
% - http://www.les.inf.puc-rio.br/opus/docs/pubs/2013/2013-02.pdf
% - http://www.les.inf.puc-rio.br/opus/docs/pubs/2012/2012-16.pdf
% - http://www.les.inf.puc-rio.br/opus/docs/pubs/2012/2012-16.pdf

The case study is an empirical research method and a well defined approach to investigate a contemporary phenomenon in its real context. [Refs 5.1 do Whohlin ] This method was chosen because we wanted to get a trusted and reproducible evaluation for handling modularity in software systems using delta oriented programming.



Eight researchers were involved in this investigation. In order to reduce researchers’ bias while performing this study, we decided to separete developers in two groups according to their experiences in software development. We point out that nobody one have had experience with delta oriented programming before.

as  showed in table ??



\subsection{Case Study: \texttt{IRIS} email client}

%
% discutir um pouco da arquitetura
% quais features foram implementadas inicialmente
% linhas de codigo e outras metricas.
%

% verificar esse paper sobre a definição de interação entre features: H. Cho, K. Lee and K. C. Kang. Feature Relation and Dependency Management: Aspect-Oriented Approach, In SPLC 2008, p. 3-11

% verificar a seção 2.2 do paper sobre FN-email que está no meu diretório particular




\subsection{First Sprint: extract product line}

\subsubsection{Second Sprint: Initial evolution of the product line}

% quais features foram introduzidas
% caracterizar o cenario de acordo com o trabalho da Lais Neves em gpce 2011

\subsubsection{Third Sprint: Final evolution of the product line}


% quais features foram introduzidas
% caracterizar o cenario de acordo com o trabalho da Lais Neves em gpce 2011


\subsection{Metrics and Tools}

% ok, acho que aqui poderiamos discutir nossas metricas de interesse.
% acho que poderiamos focar em acoplamento (ver se Lattix oferece algo),
% estabilidade do codigo, analise de impacto considerando artefatos alterados, artefatos adicionados e
% artefatos removidos. Lembrar que temos interesse em aspectos de configuracao: discussao sobre
% a decomposicao entre Persistencia e AddressBook (acho que com AOP, ficou mais modular, mais facil
% de configurar, etc.)

% descrever tambem as ferramentas e procedimentos usados na medicao. 
