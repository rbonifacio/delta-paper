\vspace{-6pt}

\section{Introduction}

There is a growing interest in the research and 
development of variant intensive software systems (such as
software product lines~\cite{}), leading to the design of several techniques for
modeling, modularising, implementing,
and reasoning about software variability. That is the case of
Delta-Oriented programming,  a prominent technique for implementing
variant systems that has evolved from an experimental,
prototypical based approach~\cite{schaeger-splc2010}
for dealing with software variability composition 
to a full fledged Java implementation~\cite{koscielny-ppj14}.

In the context of variant intensive software systems,
modularity is recognized as a fundamental property, and the
relation between modularity and variant systems (in general)
is clearly discussed within the Baldwin and Clark theory~\cite{dr-book}, which states that ``\emph{a modular 
design is a fundamental property that allows a designer to experiment 
with different alternatives}''.  Therefore, based on a modular design supported
by established design rules, developers should be able to evolve a
software product line (SPL) according to software engineering
design principles (e.g., the open-closed and principles). Nevertheless, Neves et al. describe that
evolving software product lines is a challenging and error prone task (even for
small product lines)~\cite{neves-gpce2011}, because it requires deeper impact analysis and
the involvement of both domain and product engineers to reason about
the consistence among SPL assets that represent and relate the
variability space (usually described using feature models) to the solution space (i.e.,
detailed requirements and design, source code, test cases, test scripts, and so on).

It is usually assumed that Delta-Oriented programming improves the modular decomposition 
of features. For instance, Schaefer et al. compared Delta-Oriented programming with Feature-Oriented
programming~\cite{batory-icse2003}, concluding that the first simplifies the evolution of
SPLs. In addition, a recent study compared two implementations of a
text editor product line: one implementation developed using DeltaJ 1.5 (a Java 5 implementation
of Delta-Oriented programming) and another implementation developed
using a plugin based architecture on top of the Eclipse Rich Client Platform. The authors
conclude that Delta-Oriented programming is more intuitive and simplifies both
product line implementation and configurability, by reducing the necessary code base.   
However, the involved trade-offs of using Delta-Oriented programming are not full understood.
In particular because (a) the first study is based on qualitative assessments and  lacks the use of a
rigorous approach to support its claims, and (b) the second study focuses on building a product line
from scratch and investigates redundancy and program comprehension.
Although these perspectives relate to modularity,  they do not investigate other
characteristics of a modular design--- which should support independent development
and flexibility to evolve. 

In this paper we first investigate the use of Delta-Programming to extract a
SPL from an existing project. This is a typical approach for
software product line engineering that reduces the risks of
product line adoption. We then investigate how Delta-Oriented
programming accommodate the evolution of a software
product line with respect to typical evolutionary scenarios
of a product line~\cite{neves-gpce2011}. Therefore, the contributions
of this paper are three fold {\ldots}

\vspace{-6pt}
